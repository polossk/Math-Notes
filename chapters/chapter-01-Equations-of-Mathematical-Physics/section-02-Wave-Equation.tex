\section{波动方程}
%==============================================================================%
%------------------------------------------------------------------------------%
%==============================================================================%
\paragraph{Ex 3.}
用特征线法求解 Cauchy 问题:
\[ \begin{cases}
u_t + 2u_x + u = xt, & t > 0, -\infty < x < \infty \\
u|_{t=0} = 2 - x, & -\infty < x < \infty
\end{cases} \]

\solution
易知特征线 $\diff x /\diff t = 2 - x$, 即 $x = x(t, c) = 2t + c$.

沿特征线方向, 有
\[ \begin{aligned}
\frac{\diff u(x(t, c), t)}{\diff t} &= xt - u = 2t^2 + ct - u(x(t, c), t) \\
u(c, 0) &= 2 - c;
\end{aligned} \]
原方程可以化简为:
\[ \begin{aligned}
\implies & \me^t u(x(t, c), t) - (2 - t) = \int_0^t (2t^2+ct)\diff t =
  \frac23 t^3 + \frac{c}{2} t^2 \\
\implies & u(x(t,c),t) = (2+c)\left(\frac23 t^3+\frac{c}{2}t^2\right)\me^{-t} \\
\implies & u(x,t) = \left[(x - 2t+ 2) + \frac23 t^3 + \frac{x-2t}{2}t^2 \right]
  \me^{-t} \\
\end{aligned} \]
\qed

%==============================================================================%
%------------------------------------------------------------------------------%
%==============================================================================%
\paragraph{Ex 9.}
求解并证明初值问题:
\[ \begin{cases}
u_{tt} - a^2 u_{xx} + cu = f(x, t), & -\infty < x < \infty, t > 0 \\
u|_{t=0} = \phi(x), & -\infty < x < \infty \\
u_t|_{t=0} = \psi(x), & -\infty < x < \infty
\end{cases} \]
解的唯一性, 其中 $c$ 为常数.

%==============================================================================%
%------------------------------------------------------------------------------%
%==============================================================================%
\paragraph{Ex 13.}
证明以下 Cauchy 问题
\[ \begin{cases}
\dfrac{\partial^2 u}{\partial t^2} - a^2 \dfrac{\partial^2 u}{\partial x^2}
 + b(x, t)\dfrac{\partial u}{\partial x} + c(x, t)\dfrac{\partial u}{\partial t}
 = f(x, t), & -\infty < x < \infty, t > 0 \\ 
u|_{t=0} = \phi(x), u_t|_{t=0} = \psi(x), & -\infty < x < \infty
\end{cases} \]
解的唯一性, 其中 $b(x, t)$, $c(x, t)$ 都是有解连续函数.

%==============================================================================%
%------------------------------------------------------------------------------%
%==============================================================================%
\paragraph{Ex 19.}
求解三维波动方程的 Cauchy 问题:
\[ \begin{cases}
u_{tt} = a^2 (u_{xx} + u_{yy} + u_{zz}), \\
u|_{t=0} = 0, \\
u_t|_{t=0} = x^3 + y^2 z
\end{cases} \]

%==============================================================================%
%------------------------------------------------------------------------------%
%==============================================================================%
\paragraph{Ex 22.}
求特征值问题的特征函数:
\[ \begin{cases}
X''(x) + \lambda X(x) = 0, & 0 < x < l \\
X'(0) = X(l) = 0
\end{cases} \]

%==============================================================================%
%------------------------------------------------------------------------------%
%==============================================================================%
\paragraph{令:}
\[
Lu = \frac{\partial^2 u}{\partial t^2} - a^2 \frac{\partial^2 u}{\partial x^2}
\]
\[ Q = \{ (x, t) | 0 < x < l, t > 0 \} \]

\paragraph{Ex 23.}
用分离变量法求解定解问题:
\[ \begin{cases}
Lu = 0, (x, t) \in Q \\
u|_{x=0} = u_x|_{x=l} = 0, t \geq 0 \\
u|_{t=0} = \sin^2 \dfrac{\pi x}{l}, u_t|_{t=0} = x(l - x), 0 \leq x \leq l
\end{cases} \]

%==============================================================================%
%------------------------------------------------------------------------------%
%==============================================================================%
\paragraph{Ex 25.}
设 $u(x, t)$ 适合定解问题:
\[ \begin{cases}
Lu = 0, (x, t) \in Q \\
-\dfrac{\partial u}{\partial x} + \alpha u|_{x=0} = \mu_1(t), & t \geq 0 \\
\dfrac{\partial u}{\partial x} + \beta u|_{x=l} = \mu(t), & t \geq 0 \\
u|_{t=0} = \phi(x), u_t|_{t=0} = \psi(x), & 0 \leq x \leq l
\end{cases} \]
试引进辅助函数, 把边条件齐次化, 设: (a) $\alpha > 0$, $\beta > 0$; (b)
$\alpha = \beta = 0$.

%==============================================================================%
%------------------------------------------------------------------------------%
%==============================================================================%
\paragraph{Ex 26.}
用分离变量法求解下列定解问题:
\[ \begin{cases}
Lu = A, (x, t) \in Q \\
u|_{x=0} = 0, u|_{x=l} = B, t \geq 0 \\
u|_{t=0} = \dfrac{Bx}{l}, u_t|_{t=0} = 0, 0 \leq x \leq l
\end{cases} \]

%==============================================================================%
%------------------------------------------------------------------------------%
%==============================================================================%
\paragraph{Ex 28.}
用能量不等式证明一维波动方程带有第三边值条件的初边值问题解的唯一性.

%==============================================================================%
%------------------------------------------------------------------------------%
%==============================================================================%