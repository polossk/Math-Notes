\section{方程的导出和定解条件}
%==============================================================================%
%------------------------------------------------------------------------------%
%==============================================================================%
\paragraph{Ex 1.}
一根长为 $l$ 的两端固定的弦, 如在中点把弦线提起, 使中点离开平衡位置的距离为 $a$,
然后把弦线轻轻放下, 使弦作微小横振动. 试列出弦振动所满足的定解问题.

\begin{solution}
\[
\fracpartial{^2u}{t^2} - a^2 \fracpartial{^2u}{x^2} = 0,
x \in (0, 1), t > 0
\]
端点固定: $u(0, t) = u(l, t) = 0$. 初值为两线段:
\[\begin{cases}
    u(x, 0) = \frac{2a}{l} x, & x \in (0, \frac{l}{2}] \\
    u(x, 0) = -\frac{2a}{l} (x - l), & x \in (\frac{l}{2}, l)
\end{cases} \]
初值时为静止状态: $u_t(x, 0) = 0$.

综上有:
\[ \begin{cases}
\fracpartial{^2u}{t^2} - a^2 \fracpartial{^2u}{x^2} = 0,
x \in (0, 1), t > 0\\
u(0, t) = u(l, t) = 0\\
u(x, 0) = \begin{cases}
    \frac{2a}{l} x, & x \in (0, \frac{l}{2}] \\
    -\frac{2a}{l} (x - l), & x \in (\frac{l}{2}, l)
\end{cases}\\
u_t(x, 0) = 0
\end{cases} \]
\qed
\end{solution}
%==============================================================================%
%------------------------------------------------------------------------------%
%==============================================================================%
\paragraph{Ex 3.}
设长度为 $l$ 的均匀弹性杆的线密度为 $\rho$, 杨氏模量为 $E$, 试列出杆的微小纵振动
方程.

\begin{solution}
微元 $(x, x + \Delta x)$ 的两端有伸长量 $u_t(x, t)$,
$u_t(x + \Delta x, t)$. 不妨设杆的横截面为 $S = S(x)$, 则其在 $x$ 处所受拉力
$F_x = E(x) \cdot u_t(x, t) \cdot S(x)$, 在 $x' = x + \Delta x$ 处所受拉力
$F_{x'} = E(x') \cdot u_t(x', t)\cdot S(x')$, 则有
$\sum F = F_{x'} - F_x$. 设其重心为 $\bar x$, 有
\[
a = \fracpartial{^2u}{t^2}(\bar x, t) = \frac{\sum F}{m},
\]
又由于 $m = \rho V = m \rho (S(x) x)$, 进一步化简得
\begin{align*}
&\quad\Delta x \fracpartial{^2u}{t^2}(\bar x, t) \rho(\bar x) S(\bar x) \\
&= E(x + \Delta x) u_t(x + \Delta x, t) S(x + \Delta x) \\
&\quad- E(x) u_t(x, t) S(x)
\end{align*}
由中值定理可知, $\forall \xi \in (x, x + \Delta x)$, 使得
\[
\Delta x \fracpartial{^2u}{t^2}(\bar x, t) \rho(\bar x) S(\bar x)
= \fracpartials{x}\left( E(\xi) u_x(\xi, t) S(\xi) \right) \Delta x
\]
消去 $\Delta x$,
\[
\fracpartial{^2u}{t^2}(\bar x, t) \rho(\bar x) S(\bar x)
= \fracpartials{x}\left( E(\xi) u_x(\xi, t) S(\xi) \right)
\]
当 $\Delta x \to 0$ 时, 有
\[
\fracpartials{t}\left( \rho(x)S(x) \fracpartial{u}{t}(x, t) \right)
= \fracpartials{x}\left( E(x) S(x) u_x(x, t)\right)
\]
又 $S(x)$ 恒为常数, 进一步化简得
\[
\fracpartials{t}\left( \rho(x u_t(x, t) \right)
= \fracpartials{x}\left( E(x) u_x(x, t) \right).
\]
\qed
\end{solution}
%==============================================================================%
%------------------------------------------------------------------------------%
%==============================================================================%
\paragraph{Ex 5.}
试证明圆锥形杆的微小纵振动方程是
\[
\rho \left(1 - \frac{x}{h}\right)^2 u_{tt}
= E\fracpartials{x}\left[ \left(1 - \frac{x}{h}\right)^2 u_x \right]
\]
其中 $h$ 是圆锥的高, $\rho$, $E$ 分别表示它的密度与杨氏模量, 且 $\rho$, $E$ 为常
数.

\begin{solution}
由上一题可知:
\[
\fracpartials{t}\left( \rho(x)S(x) u_t \right)
= \fracpartials{x}\left( E(x) S(x) u_x \right)
\]
此时, $S(x) = S_0\left(1 - \frac{x}{h}\right)^2$, $S_0$ 表示圆锥底的面积, 代入上
式得
\[
\fracpartials{t}\left( \rho(x) S_0\left(1 - \frac{x}{h}\right)^2 u_t \right)
= \fracpartials{x}\left( E(x) S_0\left(1 - \frac{x}{h}\right)^2 u_x \right)
\]
整理得到
\[
\rho \left(1 - \frac{x}{h}\right)^2 u_{tt}
= E\fracpartials{x}\left[ \left(1 - \frac{x}{h}\right)^2 u_x \right]
\]
\qed
\end{solution}
%==============================================================================%
%------------------------------------------------------------------------------%
%==============================================================================%
\paragraph{Ex 7.}
为了推断地球的年龄, 曾有人设想以下一个模型: 假设地球是由古时候一团炽热的岩浆逐渐
冷却而成的, 岩浆温度为 $1200\ ^\circ \text{C}$, 表面温度 $0\ ^\circ \text{C}$,
$a^2 = \frac{k}{cp} = 6 \times 10^{-7} \text{m}^2/\text{s}$, 试列出地球冷却这个
热传导过程所满足的定解问题.

\begin{solution}
建立球坐标系 $(r, \theta, \phi)$:
\[ \left\{ \begin{aligned}
r^2 &= x^2 + y^2 + z^2 \\
\theta &= \arctan \frac{y}{x} \\
\phi &= \arctan \frac{z}{r}
\end{aligned} \right. \]
\begin{align*}
&\fracpartial{u}{x} = \frac{x}{r} \fracpartial{u}{r}, 
\fracpartial{u}{y} = \frac{y}{r} \fracpartial{u}{r}, 
\fracpartial{u}{z} = \frac{z}{r} \fracpartial{u}{r} \\
& \Delta u = \fracpartial{^2u}{r^2}
  + \frac{2}{r} \fracpartial{u}{r} \\
& \fracpartial{u}{t} - a^2 \left(
  \fracpartial{^2u}{r^2} + \frac{2}{r} \fracpartial{u}{r}
  \right) = 0, \\
& r \geq 0, \theta \in [0, 2\pi],
  \phi \in [-\frac{\pi}{2}, \frac{\pi}{2}], (r, \theta, \phi) \in \Omega
\end{align*}
注意到初始条件 $u|_{t=0} = 1200$, $u|_{\partial \Omega} = 0$, 联立方程即得:
\[ \left\{ \begin{aligned} 
&\fracpartial{u}{t} - a^2 \left(
    \fracpartial{^2u}{r^2} + \frac{2}{r} \fracpartial{u}{r}
    \right) = 0 \\
&u|_{t=0} = 1200 \\
&u|_{\partial \Omega} = 0
\end{aligned} \right. \]
\qed
\end{solution}
%==============================================================================%
%------------------------------------------------------------------------------%
%==============================================================================%
\paragraph{Ex 10.}
一条从西向东无穷延伸的传送带, 运转速度为 $a$, 开始运转时传送带上空无一物, 然后在
带的起点上通过一升降机源源不断地以 $A (1 + \sin \omega t)(\text{kg})$ 的方式向传
送带加煤, 试列出在煤的传输过程中, 煤的质量分布所满足的微分方程和定解条件
({\bf 提示:} 煤的传输适合质量守恒定律).

\begin{solution}
设 $u = u(x, t)$ 表示其质量分布:
\[
\fracpartial{u}{t} + a\fracpartial{u}{x} = 0, x > 0, t > 0
\]
添加初值条件 $u(x, 0) = 0$, $x \geq 0$, $u(0, t) = A(1+\sin t)$, $t\geq 0$, 得到
方程
\[ \left\{ \begin{aligned} 
&\fracpartial{u}{t} + a\fracpartial{u}{x} = 0,
    x \geq 0, t \geq 0 \\
&u(x, 0) = 0 \\
&u(0, t) = A(1+\sin t)
\end{aligned} \right. \]
\qed
\end{solution}
%==============================================================================%
%------------------------------------------------------------------------------%
%==============================================================================%
\paragraph{Ex 11.}
写出连接平面上两点 A, B 的短程线所满足的变分问题. 若 A 的坐标为 $(0, 0)$, B 的坐
标为 $(3, 5)$, 试求出该变分问题的解.

\begin{solution}
设 $AB$ 的方程为 $y = f(x)$, $x \in [a, b]$, 则 $AB$ 的距离为
\[
|\bar{AB}| = \diff\ (f)|_A^B = \int_{x_A}^{x_B} \sqrt{1 + f'^2(x)} \diff x
\]
其中 $x_A$, $x_B$ 分别是点 $A$ 与点 $B$ 的横坐标.

变分问题为 
\[ \diff f^* = \min_{f \in \mathcal M} \diff f \]
其中 $\mathcal M$ 是函数集合
$\mathcal M = \{f | f \in C^1[a, b], f(x_{A, B}) = y_{A, B}\}$.

考查 $\forall v \in \mathcal M$, 有 $\forall \epsilon \in \mreal$, 满足
\[ J(\epsilon) = \diff\ (f^* + \epsilon v) \geq \diff\ (f^*) = J(0) \]
又由于
\[
J(\epsilon) = \int_b^a \sqrt{1 + \left(f^*_x + \epsilon v_x\right)^2} \diff x
\]
对 $J(\cdot)$ 求导, 得:
\begin{align*}
\implies J'(\epsilon)
&= \int_b^a \frac{1}{2} \cdot
  \frac{2 (f^*_x + \epsilon v_x) v_x}{1 + \left(f^*_x + \epsilon v_x\right)^2}
  \diff x \\
J'(0) &= \int_b^a
  \frac{f^*_x v_x}{\sqrt{1 + (f^*_x)^2}} \diff x \\
&= \int_b^a
  \left[ \left( \frac{f^*_x}{\sqrt{1 + (f^*_x)^2}} \right)_x \diff x - v \right]
  = 0
\end{align*}
由于 $v$ 为任意量, 故有
\[
\fracpartials{x}\left( \frac{f_x}{\sqrt{1 + f_x^2}} \right) = 0
\implies f_x = C \cdot \sqrt{1 + f_x^2}
\]
其中 $C$ 为常数. 此时有 $f_x = \tilde c$, 即最优解 $f^*$ 满足
$f^* = \tilde c x + b$ 的形式, 代入 $A(0, 0)$, $B(3, 5)$, 得 $f^* = (5 / 3) x$.
\qed
\end{solution}
%==============================================================================%
%------------------------------------------------------------------------------%
%==============================================================================%
\paragraph{Ex 14.}
设
\begin{align*}
J(v) &= \frac{1}{2} \int_{\Omega} (|\nabla v|^2 + v^2)\diff x
+ \frac{1}{2} \int_{\partial \Omega} \alpha(x) v^2 \diff s \\
&\quad- \int_{\Omega} fv \diff x - \int_{\partial \Omega} gv \diff s
\end{align*}
其中 $\alpha(x) \geq 0$. 考虑以下三个问题:

问题 I(变分问题): 求 $u \in M = C^1(\bar \Omega)$, 使得
\[ J(u)= \min_{v \in M} J(v). \]

问题 II: 求 $u \in M = C^1(\bar \Omega)$, 使得它对任意 $v \in M$, 都满足
\[
\int_{\Omega} (\nabla u \cdot \nabla v + u \cdot v - fv)\diff x
+ \int_{\partial \Omega} (\alpha(x) uv-gv )\diff s = 0.
\]

问题 III(第三边值问题): 求 $u \in C^2 \cap C^1(\bar \Omega)$, 满足以下边值问题
\[ \begin{cases}
- \Delta u + u = f, & x \in \Omega \\
\fracpartial{u}{n} + \alpha (x) u = g, & x \in \partial \Omega
\end{cases} \]

(1) 证明问题 I 与问题 II 等价.
(2) 当 $u \in C^2 \cap C^1(\bar \Omega)$ 时, 证明问题 I, II, III 等价.

\begin{solproof}[1]
(I) $\Rightarrow$ (II) (充分性):
设 $u$ 为最小值点, $\forall v \in M$, $\forall \epsilon \in \mreal$,
$\phi(\epsilon) = J(u + \epsilon v)$, $\phi'(0) = 0$, 即
\begin{align*}
\phi(\epsilon) &= J(u + \epsilon v) \\
&= \frac12 \int_\Omega 
    \left[ |\nabla u + \epsilon \nabla v|^2 + (u + \epsilon v)^2 \right]
    \diff x \\
&\quad+ \frac12 \int_{\partial\Omega} \alpha(x) (u + \epsilon v)^2 \diff s \\
&\quad- \int_\Omega f(u + \epsilon v) \diff x
    - \int_{\partial\Omega} g(u + \epsilon v) \diff s
\end{align*}
代入求导可得
\begin{align*}
\phi'(0) &= J(u) = \frac12\cdot 2\cdot \int_\Omega
    \left( \nabla u \cdot \nabla v + uv \right) \diff x \\
&\quad+ \frac12\cdot 2\cdot \int_{\partial\Omega} \alpha(x)uv \diff s \\
&\quad- \int_\Omega fv\diff x - \int_{\partial\Omega} gv\diff s \\
&= 0
\end{align*}
化简
\begin{align*}
&\int_\Omega \left( \nabla u \cdot \nabla v + uv -fv \right) \diff x \\
&\quad+ \int_{\partial\Omega} \left( \alpha(x)uv - gv \right) \diff s = 0
\end{align*}
即证 (I) $\Rightarrow$ (II).

\noindent (I) $\Leftarrow$ (II) (必要性):
同时假设 $u$ 满足 (II), $\forall v \in M$, $J(v) = J(u + v - u)$, 即
\begin{align*}
J(v) &= \frac12 \int_\Omega \left( |\nabla u|^2 + u^2 \right) \diff x \\
&\quad+ \int_\Omega\left[|\nabla u\cdot(\nabla v-\nabla u)|+u(v-u)\right]
    \diff x \\
&\quad+ \frac12 \int_\Omega \left( |\nabla (v - u)|^2 + (v - u)^2 \right)
    \diff x \\
&\quad- \int_\Omega fu \diff x - \int_\Omega f(v - u) \diff x \\
&\quad+ \frac12 \int_{\partial\Omega} \alpha(x) u^2 \diff s
    + \int_{\partial\Omega} \alpha(x) u(v - u) \diff s \\
&\quad+ \frac12 \int_{\partial\Omega} \alpha(x) (v - u)^2 \diff s \\
&\quad- \int_{\partial\Omega} gu \diff s - \int_{\partial\Omega} g(v - u)
    \diff s \\
&= J(u) + \frac12 \int_{\partial\Omega} \alpha(x) (v - u)^2 \diff s \\
&\quad+ \int_\Omega
    \left[ |\nabla u \cdot (\nabla v - \nabla u)| + u(v - u) \right] \diff x
\end{align*}
又由于
\[ \begin{equlist}
&\int_\Omega\left[|\nabla u\cdot(\nabla v-\nabla u)|+u(v-u)\right]\diff x
    \geq 0 \\
&\phantom{~~~~~~~~}(\text{平方项}) \\
&\int_{\partial\Omega} \alpha(x) (v - u)^2 \diff s \geq 0 \\
&\phantom{~~~~~~~~}(\text{平方项} \times \text{正数}) \\
\end{equlist} \]
即证 $J(v) \geq J(u)$.
\qed
\end{solproof}
%==============================================================================%
%------------------------------------------------------------------------------%
%==============================================================================%
\begin{solproof}[2]
(II) $\Rightarrow$ (III) (充分性):
对 $\forall v \in C_0^\infty(\Omega) \subset M$, 若 (II) 满足, 则有:
\begin{align*}
& \int_\Omega\left[ \nabla u\cdot\nabla v + uv - fv \right] \diff x = 0 \\
\implies & \int_\Omega (-\Delta u + u - f) v \diff x = 0 \\
\implies & -\Delta u + u - f = 0
\end{align*}
同时
\begin{align*}
& \int_{\partial\Omega}\left[ \alpha(x) uv - gv \right] \diff s = 0 \\
\implies & \int_{\partial\Omega}
  (\fracpartial{u}{n} + \alpha u - g) v \diff s = 0 \\
\implies & \fracpartial{u}{n} + \alpha u - g = 0
\end{align*}
即证.

\noindent (II) $\Leftarrow$ (III) (必要性):
对式子 $-\Delta u + u = f$ 两边同乘 $v$, $\forall v \in M$, 有
\[
-\Delta uv + uv = fv
\]
积分得
\[
\int_\Omega -\Delta uv + uv \diff x = \int_\Omega fv \diff x
\]
同理, 对式子 $u_n + \alpha(x) u = g$ 两边同乘 $v$ 之后积分, 有
\[
\int_{\partial\Omega} \fracpartial{u}{n}v \diff s
  + \int_{\partial\Omega} \alpha(x)uv \diff s
  = \int_{\partial\Omega} gv \diff s
\]
累加合并即得其解.
\qed
\end{solproof}
%==============================================================================%
%------------------------------------------------------------------------------%
%==============================================================================%
\endinput
