\section{方程的导出和定解条件}
\paragraph{Ex 1.}
% 23456789----------0123456789----------0123456789----------0123456789----------
一根长为 $l$ 的两端固定的弦, 如在中点把弦线提起, 使中点离开平衡位置的距离为 $a$,
然后把弦线轻轻放下, 使弦作微小横振动. 试列出弦振动所满足的定解问题.

\paragraph{Ex 3.}
% 23456789----------0123456789----------0123456789----------0123456789----------
设长度为 $l$ 的均匀弹性杆的线密度为 $\rho$, 杨氏模量为 $E$, 试列出杆的微小纵振动
方程.

\paragraph{Ex 5.}
% 23456789----------0123456789----------0123456789----------0123456789----------
试证明圆锥形杆的微小纵振动方程是
\[
\rho (1 - \frac{x}{h})^2 \frac{\partial^2 u}{\partial t^2}
= E \frac{\partial}{\partial x}
[(1 - \frac{x}{h})^2 \frac{\partial u}{\partial x} ]
\]
其中 $h$ 是圆锥的高, $\rho$, $E$ 分别是它的密度与杨氏模量, 且 $\rho$, $E$ 为常数.

\paragraph{Ex 7.}
% 23456789----------0123456789----------0123456789----------0123456789----------
为了推断地球的年龄, 曾有人设想以下一个模型: 假设地球是由古时候一团炽热的岩浆逐渐
冷却而成的, 岩浆温度为 $1200\ ^\circ \text{C}$, 表面温度 $0\ ^\circ \text{C}$,
$a^2 = \frac{k}{cp} = 6 \times 10^{-7} \text{m}^2/\text{s}$, 试列出地球冷却这个
热传导过程所满足的定解问题.

\paragraph{Ex 10.}
% 23456789----------0123456789----------0123456789----------0123456789----------
一条从西向东无穷延伸的传送带, 运转速度为 $a$, 开始运转时传送带上空无一物, 然后在
带的起点上通过一升降机源源不断地以 $A (1 + \sin \omega t)(\text{kg})$ 的方式向传
送带加煤, 试列出在煤的传输过程中, 煤的质量分布所满足的微分方程和定解条件
({\bf 提示:} 煤的传输适合质量守恒定律).


\paragraph{Ex 11.}
% 23456789----------0123456789----------0123456789----------0123456789----------
写出连接平面上两点 A, B 的短程线所满足的变分问题. 若 A 的坐标为 $(0, 0)$, B 的坐
标为 $(3, 5)$, 试求出该变分问题的解.

\paragraph{Ex 14.}
% 23456789----------0123456789----------0123456789----------0123456789----------
设
\[
J(v) = \frac{1}{2} \int_{\Omega} (|\nabla v|^2 + v^2)\diff x
+ \frac{1}{2} \int_{\partial \Omega} \alpha(x) v^2 \diff s
- \int_{\Omega} fv \diff x - \int_{\partial \Omega} gv \diff s
\]
其中 $\alpha(x) \geq 0$. 考虑以下三个问题:

问题 I(变分问题): 求 $u \in M = C^1(\bar \Omega)$, 使得
\[
J(u)= \min_{v \in M} J(v).
\]

问题 II: 求 $u \in M = C^1(\bar \Omega)$, 使得它对任意 $v \in M$, 都满足
\[
\int_{\Omega} (\nabla u \cdot \nabla v + u \cdot v - fv)\diff x
+ \int_{\partial \Omega} (\alpha(x) uv-gv )\diff s = 0.
\]
% 23456789----------0123456789----------0123456789----------0123456789----------
问题 III(第三边值问题): 求 $u \in C^2 \cap C^1(\bar \Omega)$, 满足以下边值问题
\[
\begin{cases}
- \Delta u + u = f, & x \in \Omega \\
\frac{\partial u}{\partial n} + \alpha (x) u = g, & x \in \partial \Omega
\end{cases}
\]

(1) 证明问题 I 与问题 II 等价.
(2) 当 $u \in C^2 \cap C^1(\bar \Omega)$ 时, 证明问题 I, II, III 等价.

\endinput