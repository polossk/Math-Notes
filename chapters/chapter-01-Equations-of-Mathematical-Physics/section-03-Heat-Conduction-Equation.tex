\section{热传导方程}
%==============================================================================%
%------------------------------------------------------------------------------%
%==============================================================================%
\paragraph{Ex 1.}
按定义求函数的 Fourier 变式
\[
f(x) = \begin{cases}0 & (\abs x > a) \\ \abs x & (\abs x \leq a) \end{cases}
\]

\solution
\[ \begin{aligned}
\hat f(\lambda) &= \frac{1}{\sqrt{2\pi}} \int_{-a}^a
 \abs{x} \me^{-\mi \lambda x} \diff x \\
&= \frac{1}{\sqrt{2\pi}} \mclosure{
	\int_0^a x \me^{-\mi \lambda x} \diff x
	-\int_{-a}^a x \me^{-\mi \lambda x} \diff x
} \\
&= \frac{2}{\sqrt{2\pi}} \mclosure{
	\frac{1}{\lambda^2} \cos (\lambda a) - \frac{1}{\lambda^2}
	+\frac{a}{\lambda} \sin (\lambda a)
} \\
\implies \hat f(\lambda) &= \sqrt{\frac{2}{\pi}} \mclosure{
	\frac{a}{\lambda} \sin (\lambda a) + \frac{1}{\lambda^2} \cos (\lambda a)
	- \frac{1}{\lambda^2}
}
\end{aligned} \]
\qed


%==============================================================================%
%------------------------------------------------------------------------------%
%==============================================================================%
\paragraph{Ex 2.}
利用 Fourier 变换的性质函数的 Fourier 变式
\[
f(x)=\begin{cases}\me^{\mi \lambda_0 x}&(\abs x< L)\\0&(\abs x\geq L)\end{cases}
\]

\solution
\[ \begin{aligned}
\hat f(\lambda) &= \frac{1}{\sqrt{2\pi}} \int_{-L}^L
 \me^{\mi (\lambda_0 - \lambda) x} \diff x \\
&= \frac{1}{\sqrt{2\pi}} \frac{1}{\mi (\lambda_0 - \lambda)}  \mclosure{
	\exp \mclosurebrace{\mi (\lambda_0 - \lambda) L}
	-\exp \mclosurebrace{-\mi (\lambda_0 - \lambda) L}
} \\
&= \sqrt{\frac{2}{\pi}} \frac{\sin(\lambda_0 - \lambda) L}{\lambda_0 - \lambda}
\end{aligned} \]
\qed

%==============================================================================%
%------------------------------------------------------------------------------%
%==============================================================================%
\paragraph{Ex 3.}
求函数的 Fourier 逆变换
\[ f(\lambda) = \me^{-a^2 \lambda^2 t},\ (t > 0 \text{为参数}) \]

\solution
\[ \begin{aligned}
f(x) &= \lim_{N \to \infty} \frac{1}{\sqrt{2\pi}} \cdot
 \int_{-N}^{N} \me^{\mi \lambda x - a^2 \lambda^2 t} \diff \lambda \\
&= \lim_{N \to \infty} \frac{1}{\sqrt{2\pi}} \mclosuresquare{
	\int_{-N}^{N} \me^{\mi \lambda x} \diff \lambda
	+ \int_{-N}^{N} \me^{-a^2 \lambda^2 t} \diff \lambda
} \\
&= \frac{1}{a\sqrt{2t}} \me^{-\frac{x^2}{4 a^2 t}}
\end{aligned} \]
\qed

%==============================================================================%
%------------------------------------------------------------------------------%
%==============================================================================%
\paragraph{Ex 4.}
应用 Fourier 变换求解以下定解问题 \\
(1) $\begin{dcases}
\fracpartial{u}{t} - a^2 \fracpartial{^2u}{x^u} - b \fracpartial{u}{x} - cu
=f(x, t), & -\infty < x < \infty, t > 0 \\
\mfwhen{u}{t=0} = \phi(x), & -\infty < x < \infty
\end{dcases}$\\
(2) $\begin{dcases}
\fracpartial{^2u}{x^2}+\fracpartial{^2u}{y^2} = 0, &-\infty<x<\infty, y > 0 \\
\mfwhen{u}{y=0} = \phi(x), & -\infty < x < \infty, \phi(x) \text{连续有界}
\end{dcases}$\\

\solution
(1): 对 $u$ 关于 $x$ 作 Fourier 变换 $\hat u = \hat u(x, t) = (u(x, t))^{\hat{}}$
\[\implies \mequlist{
&\fracdiff{\hat u}{t} +a^2\lambda^2\hat u -\mi b\lambda\hat u -c\hat u
 =\hat f(\lambda, t) \\
&\mfwhen{\hat u}{t=0} = \hat\phi(\lambda)
}
\]
解此 ODE 得到
\[
\hat u = \me^{(-a^2\lambda^2 + \mi b \lambda + c)t} \hat\phi(\lambda)
+ \int_0^t \me^{(-a^2\lambda^2 + \mi b \lambda + c)(t - \tau)}
 \hat f(\lambda, \tau) \diff \tau
\]
\[ \begin{aligned}
u &= (\hat u)^{\check{}} \\
&= \me^{\mclosure{c-\frac{b}{4a^2}}t} \cdot \frac{1}{a\sqrt{2t}}
\int_{-\infty}^{+\infty} \me^{\frac{(x - \xi)^2}{4a^2t}} \cdot
\me^{-\frac{b}{2a^2}(x-\xi)} \cdot \phi(\xi) \diff \xi \\
&+ \int_0^{t} \frac{1}{a\sqrt{t-\tau}} \me^{\mclosure{c-\frac{b}{4a^2}}(t-\tau)}
\int_{-\infty}^{+\infty} \me^{\frac{(x - \xi)^2}{4a^2(t-\tau)}} \cdot
\me^{-\frac{b}{2a^2}(x-\xi)} \cdot f(\xi, \tau) \diff \xi \diff \tau
\end{aligned} \]
即得其解

\noindent (2): 对 $u$ 关于 $x$ 作 Fourier 变换
$\hat u = \hat u(x, t) = (u(x, t))^{\hat{}}$ 则原方程可以化为
\[ \mequlist{
& (\mi\lambda)^2 \hat u(\lambda, y)	+ \fracpartial{^2\hat u}{y^2} = 0 \\
& \mfwhen{\hat u}{y=0} = \hat \phi(\lambda)
} \implies \fracpartial{^2\hat u}{y^2} = \lambda^2\hat u(\lambda, y) = 0\]
解这个 ODE 得到 $\hat u(\lambda, y) = C \cdot \me^{-\abs{\lambda} y}$, 考察初值
条件, $\mfwhen{\hat u}{y=0} = \hat\phi(\lambda)$,
$\hat u = \hat\phi(\lambda) \cdot \me^{-\abs{\lambda}y}$. 对其进行逆 Fourier 变
换, 得
\[
u(x, y) = \sqrt{\frac{2}{\pi}} \int_{-\infty}^{+\infty} \frac{y}{y^2+(x-\xi)^2}
\phi(\xi) \diff \xi
\]

\qed

%==============================================================================%
%------------------------------------------------------------------------------%
%==============================================================================%
\paragraph{Ex 5.}
证明在 $\mathcal D'(-\infty, \infty)$ 意义下 \\
(1) $\phi(x)\delta(x) = \phi(0)\delta(x)$; \\
(2) $\phi(x)\delta'(x) = -\phi'(0)\delta(x) + \phi(0)\delta'(x)$; \\
其中 $\phi \in \mscon{^\infty(-\infty, \infty)}$

\solution
引入一试验函数 $\xi(x) \in \mathcal D(\mreal)$.

\noindent (1): 对 $\phi(x) \cdot \delta(x)$, 考察 $\phi(x) \cdot \delta(x)$ 的对
偶积
\[ \begin{aligned}
<\phi(x) \cdot \delta(x), \xi(x)> &= \int_{-\infty}^{+\infty} 
	\phi(x) \cdot \delta(x) \cdot \xi(x) \diff x \\
&= <\delta(x), \phi(x) \cdot \xi(x)> = \phi(0) \cdot \xi(0)
\end{aligned} \]
又 $\xi(0) = <\delta(x), \xi(x)>$, 故有
$\phi(0)\xi(0) = \phi(0) \cdot <\delta(x), \xi(x)>$
\[ \begin{aligned}
<\phi(x) \cdot \delta(x), \xi(x)> &= \phi(0) \cdot <\delta(x), \xi(x)> \\
&= \phi(0) \int_{-\infty}^{+\infty}\delta(x)\xi(x)\diff x \\
&= <\phi(0) \cdot \delta(x), \xi(x)>
\end{aligned} \]
即证 $\phi(x)\delta(x) = \phi(0)\delta(x)$

\noindent (2): 对 $\phi(x) \cdot \delta'(x)$, 考察其对偶积
\[ \begin{aligned}
<\phi(x) \cdot \delta'(x), \xi(x)> &= \int_{-\infty}^{+\infty} 
	\phi(x) \cdot \delta'(x) \cdot \xi(x) \diff x \\
&= <\delta'(x), \phi(x)\cdot\xi(x)> = -<\delta(x), (\phi(x)\cdot\xi(x))'> \\
&= -<\delta(x), \phi(x)\cdot\xi'(x) + \phi'(x)\cdot\xi(x)> \\
&= -<\delta(x), \phi(x)\cdot\xi'(x)> - <\delta(x), \phi'(x)\cdot\xi(x)> \\
&= -\phi(0)\cdot\xi'(0) - \phi'(0)\cdot\xi(0) \\
&= -\phi(0)<\delta(x), \xi'(x)> - \phi'(0)<\delta(x), \xi(x)> \\
&= \phi(0)<\delta(x)', \xi(x)> - \phi'(0)<\delta(x), \xi(x)> \\
&= \int_{-\infty}^{+\infty} \phi(0)\cdot\delta'(x)\cdot\xi(x) \diff x
- \int_{-\infty}^{+\infty} \phi'(0)\cdot\delta(x)\cdot\xi(x) \diff x \\
&= <\phi(0)\cdot\delta'(0) - \phi'(0)\cdot\delta(0), \xi(x)>
\end{aligned} \]
即证 $\phi(x)\delta'(x) = -\phi'(0)\delta(x) + \phi(0)\delta'(x)$
\qed

%==============================================================================%
%------------------------------------------------------------------------------%
%==============================================================================%
\paragraph{Ex 7.}
计算广义导数
\[ f(x) = \begin{cases} \sin x, & x \geq 0 \\ 0, & x < 0 \end{cases} \]

\solution
对 $\forall \phi \in \mathcal (\mreal)$ 有
\[ \begin{aligned}
<f', \phi> &= -<f, \phi'> = -\int_{-\infty}^{+\infty} f\cdot\phi'\diff x \\
&= -\int_{0}^{+\infty} \sin x\cdot\phi'(x)\diff x \\
&= -\mclosuresquare{\left.\sin x\phi(x)\right|_0^{+\infty}
+ \int_{0}^{+\infty} \cos x\phi(x) \diff x} \\
&= +\int_0^{+\infty} \cos x\phi(x) \diff x \\
\end{aligned} \]
取特殊函数 $H(x) = \begin{dcases}1,&x\geq0\\0,&x<0\end{dcases}$, 有
\[ <f',\phi> = <H(x)\cdot \cos x, \phi> \]
故有 $f' = H(x)\cdot\cos(x)$
\qed

%==============================================================================%
%------------------------------------------------------------------------------%
%==============================================================================%
\paragraph{Ex 8.}
利用函数
\[ \Phi(z) = \frac{2}{\sqrt\pi} \int_0^z \me^{-\xi^2} \diff \xi \]
求解半无界问题
\[ \begin{dcases}
u_t - a^2 u_{xx} = 0, & x > 0, t > 0 \\
u(x, 0) = 0, & x \geq 0 \\
-\fracpartial{u}{x}(0, t) = q, & t > 0, q \text{为常数}
\end{dcases} \]

\solution
令 $z = \frac{x}{2a\sqrt{t}}$, 则有
\[ \begin{aligned}
\fracpartial{\Phi}{t} - a^2 \fracpartial{^2\Phi}{x^2}
&= \fracpartial{\Phi}{z} \fracpartial{z}{t}
 - a^2 \fracpartials{x}\mclosure{\fracpartial{\Phi}{x}} \\
&= \fracpartial{\Phi}{z} \fracpartial{z}{t}
 - a^2 \fracpartials{x}\mclosure{\fracpartial{\Phi}{z}\fracpartial{z}{x}} \\
&= \Phi'(z) \mclosure{\frac{-x}{4at\sqrt{t}}}
 - a^2 \frac{-x}{2a\sqrt{t}} \mclosuresquare{
  \Phi'(z) \mclosure{-2 \cdot \frac{x}{2a\sqrt{t}}} \cdot \frac{1}{2a\sqrt{t}}
 } \\
&= \Phi'(z) \mclosure{\frac{-x}{4at\sqrt{t}}}
 - \Phi'(z) \mclosure{\frac{-x}{4at\sqrt{t}}} = 0\\
\implies \fracpartial{\Phi}{t} - a^2 \fracpartial{^2\Phi}{x^2} &= 0
\end{aligned} \]
又 $\dfracpartial{\Phi}{x} = \dfracpartial{\Phi}{z} \cdot \dfracpartial{z}{x}
= \Phi'(z) \cdot \dfrac{1}{2a\sqrt{t}}$,
$-\mfwhen{\dfracpartial{\Phi}{x}}{(0, t)}=-\Phi(0)\cdot \dfrac{1}{2a\sqrt{t}}$,
令 $v = \Phi_1 - qx$, 满足
\[ \begin{dcases}
v_t - a^2 v_{xx} = 0 \\
v(x, 0) = qx = \phi(x) \\
v_x(0, t) = 0
\end{dcases} \]
同时对 $\phi(x)$ 作偶延拓
$\bar{\phi}(x) = \begin{dcases}qx,&x\geq0\\-qx,&x<0\end{dcases}$, 故有

\[ \begin{aligned}
\bar{v}(x, t) &= \int_0^\infty \mclosuresquare{
	\Gamma(x-\xi, t) + \Gamma(x+\xi, t)
} \cdot q \cdot \xi \diff \xi \\
&= \int_0^\infty \mclosuresquare{
	\frac{1}{2a\sqrt{\pi t}} \me^{-\frac{(x-\xi)^2}{4a^2t}}
	+\frac{1}{2a\sqrt{\pi t}} \me^{-\frac{(x+\xi)^2}{4a^2t}}
} \cdot q \cdot \xi \diff \xi \\
\text{令\ }y &= \frac{x+\xi}{2a\sqrt{t}} \\
\implies \bar{v}(x, t) &= \frac{1}{2a\sqrt{\pi t}} 
	\int_{\frac{x}{2a\sqrt{t}}}^\infty \me^{-y^2}\cdot q \cdot
	\mclosure{ x - 2a\sqrt{t} y } \diff \mclosure{ x - 2a\sqrt{t} y } \\
	&\ +\frac{1}{2a\sqrt{\pi t}}
	\int_{\frac{x}{2a\sqrt{t}}}^\infty \me^{-y^2}\cdot q \cdot
	\mclosure{ 2a\sqrt{t} y - x } \diff \mclosure{ 2a\sqrt{t} y - x } \\
&= -\frac{qx}{\sqrt{\pi}} \int_{\frac{x}{2a\sqrt{t}}}^\infty \me^{-y^2} \diff y
+ 2aq\sqrt{\frac{t}{\pi}} \mclosuresquare{
	\int_{\frac{x}{2a\sqrt{t}}}^{-\frac{x}{2a\sqrt{t}}} y\me^{-y^2}\diff y
	+\int_{-\frac{x}{2a\sqrt{t}}}^\infty y\me^{-y^2}\diff y
} \\
&\ +2aq\sqrt{\frac{t}{\pi}}\int_{\frac{x}{2a\sqrt{t}}}^\infty y\me^{-y^2}\diff y
-\frac{qx}{\sqrt{\pi}} \int_{\frac{x}{2a\sqrt{t}}}^\infty \me^{-y^2} \diff y \\
&= -\frac{qx}{2}\mclosure{\Phi(-\infty) - \Phi\mclosure{\frac{x}{2a\sqrt{t}}}}
+4aq\sqrt{\frac{t}{\pi}} \int_{\frac{x}{2a\sqrt{t}}}^\infty y\me^{-y^2} \diff y
-\frac{qx}{2}\mclosure{\Phi(\infty) - \Phi\mclosure{\frac{x}{2a\sqrt{t}}}}\\
\implies \bar{v}(x, t) &= -\frac{qx}{2}\mclosuresquare{
	\Phi(\infty) + \Phi(-\infty) -2\Phi\mclosure{\frac{x}{2a\sqrt{t}}}
} + 2aq\sqrt{\frac{t}{\pi}} \me^{-\frac{x^2}{4a^2t}} \\
\implies u(x, t) &= \bar{v}(x, t) - qx
\end{aligned} \]
\qed

%==============================================================================%
%------------------------------------------------------------------------------%
%==============================================================================%
\paragraph{Ex 9.}
用分离变量法求解下列混合问题:
\[ \begin{dcases}
u_t = a^2 u_{xx} + x(l-x), & 0 < x < l, t > 0 \\
\mfwhen{u}{t=0} = \sin \dfrac{\pi}{l} x, & 0 \leq x \leq l \\
\mfwhen{u}{x=0} = 0, \mfwhen{\dfracpartial{u}{x}}{x=l} = 1, & t > 0
\end{dcases} \]

\solution
令 $v = u - x$, 方程化为
\[ \begin{equlist}
&v_t - a^2 v_{xx} = x(l - x) = f(x) \\
&\mfwhen{v}{t=0} = \sin \frac{n\pi}{l} x - x = \phi(x) \\
&\mfwhen{v}{x=0} = 0, \mfwhen{\fracpartial{v}{x}}{x=l}
= \mfwhen{\fracpartial{u}{x}}{x=l} - 1 = 0
\end{equlist} \]
分离变量, 令 $v(x, t) = X(x) \cdot T(t)$
\[
\frac{X''}{X} = \frac{T'}{a^2T} = -\lambda \implies
\begin{cases} X''(x) + \lambda X(x) = 0 \\ X(0) = X'(l) = 0
\end{cases}
\]
则有 $X_n(x) = \sin \mclosure{\dfrac{2n+1}{2l}\pi x}$, 因此有
\[ \begin{aligned}
v(x, t)&=\sum_{n=1}^{\infty}T_n(t)\cdot\sin\mclosure{\dfrac{2n+1}{2l}\pi x}\\
f(x, t)&=\sum_{n=1}^{\infty}f_n(t)\cdot\sin\mclosure{\dfrac{2n+1}{2l}\pi x}\\
\phi(x)&=\sum_{n=1}^{\infty}\phi_n(t)\cdot\sin\mclosure{\dfrac{2n+1}{2l}\pi x}\\
\end{aligned} \]
由基 $\mclosurebrace{\sin \mclosure{\dfrac{2n+1}{2l}\pi x}}_{n=1}^\infty$
在 $[0, 1]$ 上正交
\[ w_n(x) &= \sin\mclosure{\dfrac{2n+1}{2l}\pi x} \]
\[ \mequlist{
f_n(t)&=\frac 2l\int_0^l x(l-x)\cdot w_n(x) \diff x\\
\phi_n(t)&=\frac 2l\int_0^l \mclosure{\sin\frac{\pi}{l}x-x}\cdot w_n(x)\diff x
} \]
代入 $v(x, t)$ 所适定的方程组中
\[ \implies \mequlist{
&\sum_{n=1}^\infty w_n(x)\cdot T_n'(t) + \mclosure{
	\frac{2n+1}{2l}\pi a
}^2 \sum_{n=1}^\infty w_n(x)\cdot T_n(t)=\sum_{n=1}^\infty w_n(x)\cdot f_n(t) \\
&T_n(0) = \phi_n
} \]
化简, 并由正交性可知
\[ \implies \mequlist{
&T_n'(t) + \mclosure{\frac{2n+1}{2l}\pi a}^2 T_n(t) = f_n(t) \\
&T_n(0) = \phi_n
} \]
令 $\theta_n = \dfrac{2n+1}{2l}\pi$, 有
\[T_n = \phi_n \me^{-\theta_n^2a^2t}
+\int_0^t f_n(\tau) \me^{-\theta_n^2a^2(t-\tau)} \diff \tau
\]
故可知 $v(x, t) = \sum_{n=1}^\infty T_n(t) \cdot w_n(x)$, $u(x, t) = x + v(x, t)$
\qed
\endinput