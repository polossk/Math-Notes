\section{位势方程}
%==============================================================================%
%------------------------------------------------------------------------------%
%==============================================================================%
\paragraph{Ex 1.}
设 $u(x)$ 是定解问题
\[ \begin{cases}
-\Delta u + c(x) u = f(x), & x \in \Omega \\
\mfwhen{u}{\partial \Omega} = 0
\end{cases} \]
的一个解 \\
(1) 如果 $c(x) \geq C_0 > 0$, 则有估计
\[ \max_{\bar\Omega}\abs{u(x)} \leq C_0^{-1} \sup_{\Omega}\abs{f(x)} \]\\
(2) 如果 $c(x) \geq 0$ 且有解, 则
\[ \max_{\bar\Omega}\abs{u(x)} \leq M \sup_{\Omega}\abs{f(x)} \]
其中 $M$ 依赖于 $c(x)$ 的界与 $\Omega$ 的直径.
\solution
\noindent (1): 记 $Lu = -\Delta u+c(x)u$, 作辅助函数 $w(x)=\frac F{c_0}\pm u$ 得
\[ \begin{equlist}
&Lw(x) = c(x) \cdot \frac{F}{c_0} \pm f \geq F \pm f \geq 0 \\
&\mfwhen{w}{\partial\Omega} = \frac{F}{c_0} \geq 0
\end{equlist} \]
由弱极值原理得: $\mfwhen{w\geq0}{\Omega}$, 则有
\[\implies \abs{u} \leq \frac{F}{c_0} = c_0^{-1} \cdot \sup\abs{f(x)}\]
\[\implies \max_{\bar\Omega}\abs{u}\leq c_0^{-1} \cdot \sup\abs{f(x)}\]
\noindent (2): 作辅助函数 $w(x)=\frac F{2n}\mclosure{d^2-\abs{x}^2}\pm u$, 其中
$d = \sum_{x, y\in\Omega}\abs{x-y}$.
\[
Lw(x) = 2n \cdot \frac{F}{2n} + (1x)\frac{F}{2n}\mclosure{d^2-\abs{x}^2} \pm
f(x) \geq 0
\]
$\mfwhen{w}{\partial\Omega}\geq0$, 由极值原理得 $w\geq0$, 于是有
\[\abs{u}\leq\frac F{2n}\mclosure{d^2-\abs{x}^2}\leq\sup_{x\in\Omega}
\mclosure{\frac{d^2-\abs{x}^2}{2n}}\cdot F\leq\frac{d^2}{2n}F\]
故取 $M = \dfrac{d^2}{2n}$, 有:
$\max\limits_{\bar\Omega}\abs{u}\leq\dfrac{d^2}{2n}\sup\limits_\Omega\abs{f(x)}$
\[\max_{\bar\Omega}\abs{u}\leq M\sup_\Omega\abs{f(x)}\]
\qed

%==============================================================================%
%------------------------------------------------------------------------------%
%==============================================================================%
\paragraph{Ex 2.}
设 $u(x)$ 是定解问题
\[ \begin{cases}
-\Delta u + c(x) u = f(x), x \in \Omega \\
\mfwhen{\fracpartial{u}{n} + \alpha(x)u}{\Gamma_1} = \phi_1,
\mfwhen{u}{\Gamma_2} = \phi_2
\end{cases} \]
的解, 其中 $\Gamma_1 \cup \Gamma_2 = \partial\Omega$,
$\Gamma_1 \cap \Gamma_2 = \varnothing$, $\Gamma_2 \neq \varnothing$ \\
如果 $c(x) \leq 0$, $\alpha(x) \leq \alpha_0 > 0$, 则有估计
\[ \max_{\bar\Omega} \abs{u(x)} \leq C\mclosure{
	\sup_{\Omega}\abs{f}+\sup_{\Gamma_1}\abs{\phi_1}+\sup_{\Gamma_2}\abs{\phi_2}
} \]
其中常数 $C$ 依赖于 $\alpha_0$ 与 $\Omega$ 的直径.
\solution
取 $w(x) = Z(x) + \dfrac{\Phi_1}{\alpha_0} + \Phi_2 \pm u$, 其中
$Z(x)=\dfrac F{2n}\mclosure{\dfrac{1+\alpha^2}{\alpha_0^2}+\alpha^2-\abs{x}^2}$,
则有 
\[ \begin{aligned}
&-\Delta w(x) + c(x)w(x) = -\Delta Z+c(x)Z+c(x)\mclosure{
	\dfrac{\Phi_1}{\alpha_0} + \Phi_2
} \pm (-\Delta u) \pm c(x)u \\
&\quad \geq F + c(x) + \mclosure{
	\dfrac{\Phi_1}{\alpha_0} + \Phi_2
} \pm f \geq 0
\end{aligned} \]
\[
\mclosure{\fracpartial wn+\alpha(x)w}_{\Gamma_1}
=\mclosure{\fracpartial Zn+\alpha(x)Z}_{\Gamma_1}
+\alpha(x)\frac{\Phi_1}{\alpha_0} + \alpha(x)\Phi_2 \pm \phi_1 \geq 0
\]
\[
\mfwhen{w}{\Gamma_2}=\mfwhen{Z(x)}{\Gamma_2}+\frac{\Phi_1}{\alpha_0}+\Phi_2
\pm \mfwhen{u}{\Gamma_2} \geq \Phi_2 \pm \phi_2 \geq 0
\]
由弱极值原理可知, 对 $w(x)$, 其最小值在边界上取到, 即 $w(x) \geq 0$, 由此有
\[\abs{u(x)} \leq Z(x) + \frac{\Phi_1}{\alpha_0} + \Phi_2\]
于是
\[\max_{\bar\Omega} \abs{u(x)} \leq C\mclosure{
	\sup\abs f+\sup\abs{\phi_1}+\sup\abs{\phi_2}
} \]
其中
\[ C = \max\mclosurebrace{
	\frac 1{\alpha_0}, 1, \frac 1{2n}\mclosure{\frac{1+d^2}{\alpha_0} + d^2}
} \]
\qed

%==============================================================================%
%------------------------------------------------------------------------------%
%==============================================================================%
\paragraph{Ex 5.}
设 $\Omega_0$ 是三维有界区域, $\Omega = \mreal^3\textbackslash\bar\Omega_0$. 又设
$u \in C^2(\Omega) \cap C(\bar\Omega)$ 是外部问题
\[ \begin{cases}
-\Delta u + c(x) u = 0, & x \in \Omega \\
\mfwhen{u}{\partial\Omega} = \phi \\
\lim_{\abs x \to \infty} u(x) = l
\end{cases} \]
的解, 其中 $C(x) \leq 0$ 且在 $\bar\Omega$ 上局部有界, 则有估计
\[ \sup_{\Omega} \abs{u(x)} \leq \max \mclosurebrace{
	\abs l, \max_{\partial\Omega} \abs{\phi(x)}
} \]
\solution
由 $\lim_{\abs x \to \infty} u(x) = l$, 对 $\forall\epsilon>0$, $\exists a>0$,
当 $\abs{x}>a$ 时, 有 $\abs{u(x)-l}<\epsilon$.

取 $a$ 充分大, 使得 $B_a = \big\{x \big| \abs{x} < a\big\}$ 包含 $\Omega_0$, 取
$\Omega_1 = B_a \setminus \Omega_0$, 在 $\Omega_1$ 上用最大模估计
\[
\max_{\Omega} \geq \max\mclosurebrace{
	\max_{\partial\Omega_0}\abs{u}, \max_{\partial\Omega_1}\abs{u}
} \geq \max\mclosurebrace{
	\max_{\partial\Omega_0}\abs{u}, \abs{l} + \epsilon
}
\]
故当 $\epsilon \to 0$ 时, 
\[ \max_\Omega\abs{u} \geq \max\mclosurebrace{\abs{l},\max_\Omega\abs{\phi}} \]
\qed

%==============================================================================%
%------------------------------------------------------------------------------%
%==============================================================================%
\paragraph{Ex 16.}
设 $\Omega = \mclosurebrace{(x, y) | 0 < x < a, 0 < y < b}$, 用分离变量法求解稳
定的温度场 $u(x, y)$, 满足在 $x = a$, $y = b$ 上绝热, 在 $x = 0$, $y = 1$ 上温度
的值分别为 0 与 1.
\solution
列方程为
\[\begin{dcases}
-\Delta u = 0 \\
\mfwhen{\fracpartial{u}{x}}{y=b} = 0, \mfwhen{\fracpartial{u}{y}}{x=a} = 0 \\
u(0, y) = 0, u(x, 0) = 1
\end{dcases} \]
用分离变量法求解方程, 设 $u = X(x)\cdot Y(y)$ 代入 $-\Delta u=0$, 得
\[\begin{dcases}
X''(x)\cdot Y(y) + X(x)\cdot Y''(y) = 0 \\
X'(x)Y(b) = 0, X(0)Y(y) = 0 \\
X(a)Y'(0) = 0, X(x)Y(0) = 1 \\
\end{dcases} \]
进一步得
\[ \frac{X''}X = -\frac{Y''}Y \triangleq \lambda \]
对于方程
\[ \begin{dcases}
X''+\lambda X = 0 \\ X(0) = X(a) = 0
\end{dcases} \]
设其通解为: $X(x)=C_1\sin\sqrt\lambda x + C_2\cos\sqrt\lambda x$, 考虑边界条件
$C_1\sin\sqrt\lambda a = C_2 = 0$, 则可知 $\lambda=\mclosure{\frac{n\pi}a}^2$.
即
\[ X_n(x) = C \cdot \sin\frac{n\pi}a x, n = 1, 2, \ldots
\]
代入方程 $Y'(y) - \lambda Y(y) = 0$, 得
\[ Y_n(y) = A_n \me^{\frac{n\pi}a y} + B_n \me^{-\frac{n\pi}a y}
\]
考虑边界条件 $X'(x)Y(b) = 0, X(x)Y(0) = 1$
有
\[\begin{equlist}
\sum_{n=1}^\infty \mclosure{
	A_n \me^{\frac{n\pi}a y} + B_n \me^{-\frac{n\pi}a y}
} \cos\frac{n\pi}a x &= 0 \\
\sum_{n=1}^\infty \mclosure{ A_n  + B_n } \sin\frac{n\pi}a x &= 1
\end{equlist}\]
\[\implies\begin{equlist}
&A_n + B_n = \frac 2a\int_0^a \sin\frac{n\pi}a x\diff x \\
&A_n \me^{\frac{n\pi}a y} + B_n \me^{-\frac{n\pi}a y} = 0
\end{equlist}\]
\[\implies\begin{equlist}
A_n &= \frac{2\mclosuresquare{1+(-1)^{n+1}}}{n\pi\mclosure{
	1-\me^{\frac{2n\pi}{a}b}}
}\\
B_n &= -\frac{2\mclosuresquare{1+(-1)^{n+1}}}{n\pi\mclosure{
	1-\me^{\frac{2n\pi}{a} b}}
} \cdot \me^{\frac{2n\pi}{a} b}
\end{equlist}\]
于是原方程的解为
\[ u(x, y) = \sum_{n=1}^\infty \mclosure{
	A_n \me^{\frac{n\pi}a y} + B_n \me^{-\frac{n\pi}a y}
} \cdot \sin\frac{n\pi}a x \]
\qed

%==============================================================================%
%------------------------------------------------------------------------------%
%==============================================================================%
\paragraph{Ex 19.}
记 $B^{+}(R) = \mclosurebrace{(x, y)| x^2 + y^2 < R^2, y > 0}$, 求定解问题
\[\begin{dcases}
-\Delta u = f(x, y), & (x, y) \in B^{+}(R) \\
\mfwhen{u}{\partial B^{+}(R) \cap \{y>0\}} = \phi(x, y) \\
\mfwhen{\fracpartial{u}{y}}{y=0} = \psi(x, 0), & -R \leq x \leq R
\end{dcases} \]
的 Green 函数. 如果
$u \in C^1\mclosure{\overline{B^{+}}\mclosure{R}} \cap C^2\mclosure{B^{+}(R)}$
是上述问题的解, 试给出解的表达式.

\solution
Green 函数满足
\[\begin{dcases}
-\Delta G = \delta(x - \xi, y - \eta) \\
\mfwhen{G}{(x, y) \in \partial B^{+} \cap {y>0}} = 0 \\
\mfwhen{\fracpartial{G}{y}}{y=0} = 0
\end{dcases} \]
因此, 将 $G(x, y : \xi, \eta)$ 关于 $y$ 作偶延拓, 使 $G(x, y : \xi, \eta)$ 定义
于 $B^{+}$ 上.

记延拓后的函数为 $\tilde{G}(x, y : \xi, \eta)$, 则 $\tilde{G}$ 应满足
\[\begin{dcases}
-\Delta \tilde{G} = \delta(x - \xi, y - \eta) + \delta(x - \xi, -y - \eta) \\
\mfwhen{\tilde{G}}{(\partial B^{+}} = 0
\end{dcases} \]
\[ \tilde{G}(x, y : \xi, \eta)=G_0(x, y : \xi, \eta)+G_0(x, y : \xi, -\eta) \]
\[ \implies u(\xi, \eta) = \iint_\Omega \tilde{G}f\diff x\diff y
-\int_{\partial B^{+} \cap {y>0}} \fracpartial{\tilde{G}}{n}\phi \diff l \]
\qed

%==============================================================================%
%------------------------------------------------------------------------------%
%==============================================================================%
\paragraph{Ex 20.}
设 $\mreal^2_{+} = \mclosurebrace{(x, y)|-\infty<x<\infty, y>0}$, 求 Dirichlet
问题
\[\begin{cases}
\Delta u = 0, & (x, y) \in \mreal^2_{+} \\
\mfwhen{u}{y=0} = u_0(x), & -\infty < x < \infty
\end{cases} \]
的有界解, 其中
$u_0(x) = \begin{cases}1, & x\in[a, b] \\ 0, & x \notin[a, b]\end{cases}$
\solution
构造, 于上半平面 $M_0(x_0, y_0)$ 上放置一单位正电荷 $\me_0$, $M$ 关于边界
$\Gamma$ 的对称点为 $M_1(x_0, -y_0)$, 于 $M_1$ 上放置一单位负电荷 $-\me_0$,
则 $\Gamma$ 上电位相互抵消.
\[ \implies \mfwhen{\frac 1{4\pi r_{MM_0}}}{\Gamma} =
\mfwhen{\frac 1{4\pi r_{MM_1}}}{\Gamma} \]
\[ G(M, M_0) = \frac 1{4\pi r_{MM_0}} - \frac 1{4\pi r_{MM_1}} \]

\[ \begin{aligned}
\mfwhen{\fracpartial{G}{n}}{y=0} &= -\mfwhen{\fracpartial{G}{n}}{y=0}
= -\frac 1{4\pi} \cdot \mfwhen{\fracpartials{y} \mclosuresquare{
\frac 1{\sqrt{(x-x_0)^2 + (y-y_0)^2}} - \frac 1{\sqrt{(x-x_0)^2 + (y+y_0)^2}}
}}{y=0} \\
&= -\frac 1{2\pi} \mfwhen{\frac{y_0}{\mclosuresquare{
	(x-x_0)^2+(y-y_0)^2
}^{3/2}}}{y=0}
\end{aligned} \]
代入 Dirichlet 问题, 有
\[ \begin{aligned}
u(x_0, y_0) &= \frac 1{2\pi} \iint_\Omega \frac{y_0}{\mclosuresquare{
	(x-x_0)^2+y_0^2
}^{3/2}} \cdot u_0(x) \diff x \diff y \\
&= \frac 1{2\pi} \int_{-\infty}^{+\infty} \int_a^b \frac{y_0}{\mclosuresquare{
	(x-x_0)^2+y_0^2
}^{3/2}}  \diff x \diff y \\
\end{aligned} \]
\qed
\endinput
