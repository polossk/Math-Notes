% pdf 设置区
%---------------------------------纸张大小设置---------------------------------%
\usepackage{geometry}
\geometry{a4paper,left=1.91cm,right=1.91cm,top=2.05cm,bottom=2.05cm}
%------------------------------------------------------------------------------%


%----------------------------------必要库支持----------------------------------%
\usepackage{xcolor}
\usepackage{tikz}
\usepackage{layouts}
\usepackage[numbers,sort&compress]{natbib}
\usepackage{clrscode}
\usepackage{gensymb}
%------------------------------------------------------------------------------%


%--------------------------------设置标题与目录--------------------------------%
\usepackage[sf]{titlesec}
\usepackage{titletoc}
%------------------------------------------------------------------------------%


%--------------------------------添加书签超链接--------------------------------%
\usepackage[unicode=true,colorlinks=false,pdfborder={0 0 0}]{hyperref}
    % 在此处修改打开文件操作
    \hypersetup{
        bookmarks=true,         % show bookmarks bar?
        bookmarksopen=true,     % expanded all bookmark?
        pdftoolbar=true,        % show Acrobat’s toolbar?
        pdfmenubar=true,        % show Acrobat’s menu?
        pdffitwindow=true,      % window fit to page when opened
        pdfstartview={FitH},    % fits the width of the page to the window
        pdfnewwindow=true,      % links in new PDF window
    }
    % 在此处添加文章基础信息
    \hypersetup{
        pdftitle={},
        pdfauthor={polossk},
        pdfsubject={},
        pdfcreator={XeLaTeX},
        pdfproducer={XeLaTeX},
        pdfkeywords={},
    }
%------------------------------------------------------------------------------%


%---------------------------------设置字体大小---------------------------------%
\usepackage{type1cm}
% 字号与行距,统一前缀s(a.k.a size)
\newcommand{\sChuhao}{\fontsize{42pt}{63pt}\selectfont}                 % 初号, 1.5倍
\newcommand{\sYihao}{\fontsize{26pt}{36pt}\selectfont}                  % 一号, 1.4倍
\newcommand{\sErhao}{\fontsize{22pt}{28pt}\selectfont}                  % 二号, 1.25倍
\newcommand{\sXiaoer}{\fontsize{18pt}{18pt}\selectfont}                 % 小二, 单倍
\newcommand{\sSanhao}{\fontsize{16pt}{24pt}\selectfont}                 % 三号, 1.5倍
\newcommand{\sXiaosan}{\fontsize{15pt}{22pt}\selectfont}                % 小三, 1.5倍
\newcommand{\sSihao}{\fontsize{14pt}{21pt}\selectfont}                  % 四号, 1.5倍
\newcommand{\sSHalfXiaosi}{\fontsize{13pt}{19pt}\selectfont}            % 半小四, 1.5倍
\newcommand{\sHalfXiaosi}{\fontsize{13pt}{16.25pt}\selectfont}          % 半小四, 1.25倍
\newcommand{\sRealHalfXiaosi}{\fontsize{12.5pt}{16.25pt}\selectfont}    % 模板中的半小四, 1.25倍
\newcommand{\sXiaosi}{\fontsize{12pt}{14.4pt}\selectfont}               % 小四, 1.25倍
\newcommand{\sLargeWuhao}{\fontsize{11pt}{11pt}\selectfont}             % 大五, 单倍
\newcommand{\sWuhao}{\fontsize{10.5pt}{10.5pt}\selectfont}              % 五号, 单倍
\newcommand{\sXiaowu}{\fontsize{9pt}{9pt}\selectfont}                   % 小五, 单倍
%------------------------------------------------------------------------------%


%---------------------------------设置中文字体---------------------------------%
\usepackage{fontspec}
\usepackage[SlantFont,BoldFont,CJKchecksingle]{xeCJK}
\usepackage{CJKnumb}
% 使用 Adobe 字体
\newcommand\adobeSog{Adobe Song Std}
\newcommand\adobeHei{Adobe Heiti Std}
\newcommand\adobeKai{Adobe Kaiti Std}
\newcommand\adobeFag{Adobe Fangsong Std}
\newcommand\codeFont{Consolas}
% 设置字体
\defaultfontfeatures{Mapping=tex-text}
\setCJKmainfont[ItalicFont=\adobeKai, BoldFont=\adobeHei]{\adobeSog}
\setCJKsansfont[ItalicFont=\adobeKai, BoldFont=\adobeHei]{\adobeSog}
\setCJKmonofont{\codeFont}
\setmonofont{\codeFont}
% 设置字体族
\setCJKfamilyfont{song}{\adobeSog}      % 宋体
\setCJKfamilyfont{hei}{\adobeHei}       % 黑体
\setCJKfamilyfont{kai}{\adobeKai}       % 楷体
\setCJKfamilyfont{fang}{\adobeFag}      % 仿宋体
% 用于页眉学校名,特殊字体,powerby https://github.com/ecomfe/fonteditor
\setCJKfamilyfont{nwpu}{nwpuname}
% 新建字体命令,统一前缀f(a.k.a font)
\newcommand{\fSong}{\CJKfamily{song}}
\newcommand{\fHei}{\CJKfamily{hei}}
\newcommand{\fFang}{\CJKfamily{fang}}
\newcommand{\fKai}{\CJKfamily{kai}}
\newcommand{\fNWPU}{\CJKfamily{nwpu}}
%------------------------------------------------------------------------------%


%------------------------------添加插图与表格控制------------------------------%
\usepackage{graphicx}
\usepackage[font=small,labelsep=quad]{caption}
\usepackage{wrapfig}
\usepackage{multirow,makecell}
\usepackage{longtable}
\usepackage{booktabs}
\usepackage{tabularx}
\usepackage{setspace}
\captionsetup[table]{labelfont=bf,textfont=bf}
%------------------------------------------------------------------------------%


%---------------------------------添加列表控制---------------------------------%
\usepackage{enumerate}
\usepackage{enumitem}
%------------------------------------------------------------------------------%


%---------------------------------设置引用格式---------------------------------%
\renewcommand\figureautorefname{图}
\renewcommand\tableautorefname{表}
\renewcommand\equationautorefname{式}
\newcommand\myreference[1]{[\ref{#1}]}
\newcommand\eqrefe[1]{式(\ref{#1})}
% 增加 \ucite 命令使显示的引用为上标形式
\newcommand{\ucite}[1]{$^{\mbox{\scriptsize \cite{#1}}}$}
\renewcommand\arraystretch{1.4}
\renewcommand\theequation{\thesection.\arabic{equation}}
\renewcommand{\thefigure}{\thechapter-\arabic{figure}}
\renewcommand{\thetable}{\thechapter-\arabic{table}}
%------------------------------------------------------------------------------%


%--------------------------------设置定理类环境--------------------------------%
\usepackage[amsthm,thmmarks]{ntheorem}
%------------------------------------------------------------------------------%


%--------------------------设置中文段落缩进与正文版式--------------------------%
\XeTeXlinebreaklocale "zh"                      % 使用中文的换行风格
\XeTeXlinebreakskip = 0pt plus 1pt              % 调整换行逻辑的弹性大小
% \usepackage{indentfirst}                        % 段首空格设置
% \setlength{\parindent}{26pt}                    % 段首空格长度
% \setlength{\parskip}{3pt plus 1pt minus 1pt}    % 段落间距
% \renewcommand{\baselinestretch}{1.25}           % 行距
%------------------------------------------------------------------------------%

%----------------------------设置段落标题与目录格式----------------------------%
\setcounter{secnumdepth}{3}
\setcounter{tocdepth}{2}

\newcommand\chapterID[1]{第\CJKnumber{#1}章}
\renewcommand{\chaptername}{第~\CJKnumber{\thechapter}~章}
\renewcommand{\figurename}{图}
\renewcommand{\tablename}{表}
\renewcommand{\bibname}{参考文献}
\renewcommand{\contentsname}{目~录}
\newcommand{\keywords}[1]{\\ \\ \textbf{关~键~词}:#1}

\titleformat{\chapter}[hang]{\normalfont\sSanhao\filcenter\fHei\bf}%
    {\sSanhao{\chaptertitlename}}{20pt}{\sSanhao}
\titleformat{\section}[hang]{\fHei \bf \sXiaosan}%
    {\sXiaosan \thesection}{0.5em}{}{}
\titleformat{\subsection}[hang]{\fHei \bf \sSHalfXiaosi}%
    {\sSHalfXiaosi \thesubsection}{0.5em}{}{}
\titleformat{\subsubsection}[hang]{\fHei \bf}%
    {(\arabic{subsubsection})}{0.5em}{}{}   % 小标题式的subsubsection:(4) 标题

% 缩小正文中各级标题之间的缩进
\titlespacing{\chapter}{0pt}{-3ex plus .1ex minus .2ex}{0.25em}
\titlespacing{\section}{0pt}{-0.2em}{0em}
\titlespacing{\subsection}{0pt}{0.5em}{0em}
\titlespacing{\subsubsection}{0pt}{0.25em}{0pt}

% 定义目录中各级标题之间的格式以及缩进
\dottedcontents{section}[1.16cm]{}{1.8em}{5pt}
\dottedcontents{subsection}[2.00cm]{}{2.7em}{5pt}
\dottedcontents{subsubsection}[2.86cm]{}{3.4em}{5pt}
\titlecontents{chapter}[0pt]{\fHei\vspace{0.5em}}%
    {\contentsmargin{0pt}\fHei\makebox[0pt][l]{\chapterID{\thecontentslabel}}\hspace{3.8em}}%
    {\contentsmargin{0pt}\fHei}%
    {\titlerule*[.5pc]{.}\contentspage}[\vspace{0em}]

% 删除 chapter 分页
\usepackage{xpatch}
\makeatletter
\xpatchcmd{\chapter}
  {\if@openright\cleardoublepage\else\clearpage\fi}{\par\relax}
  {}{}
\makeatother
%------------------------------------------------------------------------------%


%---------------------------------设置页眉页脚---------------------------------%
\usepackage{fancyhdr}
\usepackage{fancyref}
\pagestyle{fancy}
\renewcommand{\chaptermark}[1]{\markboth{\chaptertitlename\ #1}{}}
\cfoot{}
\fancyfoot[RO, LE]{\thepage}
\fancyfoot[LO]{{\tt \artversion}}
\fancyfoot[RE]{{\it \shorttitle}}
\fancypagestyle{plain}{
    \fancyhf{}
    \fancyfoot[RO, LE]{\thepage}
    \fancyfoot[LO]{{\tt \artversion}}
    \fancyfoot[RE]{{\it \shorttitle}}
    \renewcommand{\headrulewidth}{0pt}
    \renewcommand{\footrulewidth}{0pt}
}

\renewcommand{\headrulewidth}{0.4pt}
\renewcommand{\headwidth}{\textwidth}
\renewcommand{\footrulewidth}{0pt}
%------------------------------------------------------------------------------%


%-------------------------------数学特殊符号控制-------------------------------%
\usepackage{math-symbols}
%------------------------------------------------------------------------------%


%---------------------------------添加列表控制---------------------------------%
\usepackage{enumerate}
\numberwithin{equation}{section}
\renewcommand\theequation{\thesection.\arabic{equation}}
%------------------------------------------------------------------------------%


%----------------------------------添加代码控制--------------------------------%
\usepackage{listings}
\lstset{
    basicstyle=\footnotesize\ttfamily,
    numbers=left,
    numberstyle=\tiny,
    numbersep=5pt,
    tabsize=4,
    extendedchars=true,
    breaklines=true,
    keywordstyle=\color{blue}\bfseries,
    numberstyle=\color{purple},
    commentstyle=\color[rgb]{0, 0.4, 0}\bfseries,
    stringstyle=\color{violet}\ttfamily\bfseries,
    rulesepcolor=\color{red!20!green!20!blue!20},
    showspaces=false,
    showtabs=false,
    frame=shadowbox,
    framexrightmargin=5pt,
    framexbottommargin=4pt,
    showstringspaces=false,
    escapeinside=`', %逃逸字符(1左面的键),用于显示中文
}
\renewcommand{\lstlistingname}{CODE}
\lstloadlanguages{% Check Dokumentation for further languages, page 12
    Pascal, C++, Java, Ruby, Python, Matlab, R, Haskell
}
%------------------------------------------------------------------------------%

\endinput
%这是简单的 article 的导言区设置,不能单独编译。
