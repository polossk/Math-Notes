% pdf 设置区
%---------------------------------纸张大小设置---------------------------------%
\usepackage{geometry}
\geometry{b5paper,left=2cm,right=2cm,top=2.05cm,bottom=2.05cm}
%------------------------------------------------------------------------------%


%----------------------------------必要库支持----------------------------------%
\usepackage{amsmath}
\usepackage{amssymb}
\usepackage{amsfonts}
\usepackage{mathrsfs}
\usepackage{bm}
\usepackage{xcolor}
\usepackage{tikz}
\usepackage{layouts}
\usepackage[numbers,sort&compress]{natbib}
\usepackage{clrscode}
\usepackage{gensymb}
\usepackage{varwidth}
%------------------------------------------------------------------------------%

%--------------------------------设置标题与目录--------------------------------%
\usepackage[sf]{titlesec}
\usepackage{titletoc}
%------------------------------------------------------------------------------%

%--------------------------------添加书签超链接--------------------------------%
\usepackage[unicode=true,colorlinks=false,pdfborder={0 0 0}]{hyperref}
    % 在此处修改打开文件操作
    \hypersetup{
        bookmarks=true,         % show bookmarks bar?
        pdftoolbar=true,        % show Acrobat’s toolbar?
        pdfmenubar=true,        % show Acrobat’s menu?
        pdffitwindow=true,      % window fit to page when opened
        pdfstartview={FitH},    % fits the width of the page to the window
        pdfnewwindow=true,      % links in new PDF window
    }
    % 在此处添加文章基础信息
    \hypersetup{
        pdftitle={},
        pdfauthor={shen, cfrpg},
        pdfsubject={},
        pdfcreator={XeLaTeX},
        pdfproducer={XeLaTeX},
        pdfkeywords={},
    }
%------------------------------------------------------------------------------%


%---------------------------------设置字体大小---------------------------------%
\usepackage{type1cm}
% 字号与行距,统一前缀s(a.k.a size)
\newcommand{\sChuhao}{\fontsize{42pt}{63pt}\selectfont}         % 初号, 1.5倍
\newcommand{\sYihao}{\fontsize{26pt}{36pt}\selectfont}          % 一号, 1.4倍
\newcommand{\sErhao}{\fontsize{22pt}{28pt}\selectfont}          % 二号, 1.25倍
\newcommand{\sXiaoer}{\fontsize{18pt}{18pt}\selectfont}         % 小二, 单倍
\newcommand{\sSanhao}{\fontsize{16pt}{24pt}\selectfont}         % 三号, 1.5倍
\newcommand{\sXiaosan}{\fontsize{15pt}{22pt}\selectfont}        % 小三, 1.5倍
\newcommand{\sSihao}{\fontsize{14pt}{21pt}\selectfont}          % 四号, 1.5倍
\newcommand{\sHalfXiaosi}{\fontsize{13pt}{19.5pt}\selectfont}   % 半小四, 1.5倍
\newcommand{\sXiaosi}{\fontsize{12pt}{14.4pt}\selectfont}       % 小四, 1.25倍
\newcommand{\sLargeWuhao}{\fontsize{11pt}{11pt}\selectfont}     % 大五, 单倍
\newcommand{\sWuhao}{\fontsize{10.5pt}{10.5pt}\selectfont}      % 五号, 单倍
\newcommand{\sXiaowu}{\fontsize{9pt}{9pt}\selectfont}           % 小五, 单倍
%------------------------------------------------------------------------------%


%---------------------------------设置中文字体---------------------------------%
\usepackage{fontspec}
\usepackage[SlantFont,BoldFont,CJKchecksingle]{xeCJK}
\usepackage{CJKnumb}
% 使用 Adobe 字体
\newcommand\adobeSog{Adobe Song Std}
\newcommand\adobeHei{Adobe Heiti Std}
\newcommand\adobeKai{Adobe Kaiti Std}
\newcommand\adobeFag{Adobe Fangsong Std}
\newcommand\codeFont{Consolas}
% 设置字体
\defaultfontfeatures{Mapping=tex-text}
\setCJKmainfont[ItalicFont=\adobeKai, BoldFont=\adobeHei]{\adobeSog}
\setCJKsansfont[ItalicFont=\adobeKai, BoldFont=\adobeHei]{\adobeSog}
\setCJKmonofont{\adobeSog}
\setmonofont{\codeFont}
% 设置字体族
\setCJKfamilyfont{song}{\adobeSog}      % 宋体  
\setCJKfamilyfont{hei}{\adobeHei}       % 黑体  
\setCJKfamilyfont{kai}{\adobeKai}       % 楷体  
\setCJKfamilyfont{fang}{\adobeFag}      % 仿宋体
% 用于页眉学校名,特殊字体,powerby https://github.com/ecomfe/fonteditor
\setCJKfamilyfont{nwpu}{nwpuname}
% 新建字体命令,统一前缀f(a.k.a font)
\newcommand{\fSong}{\CJKfamily{song}}
\newcommand{\fHei}{\CJKfamily{hei}}
\newcommand{\fFang}{\CJKfamily{fang}}
\newcommand{\fKai}{\CJKfamily{kai}}
\newcommand{\fNWPU}{\CJKfamily{nwpu}}
%------------------------------------------------------------------------------%


%------------------------------添加插图与表格控制------------------------------%
\usepackage{graphicx}
\usepackage[font=small,labelsep=quad]{caption}
\usepackage{wrapfig}
\usepackage{multirow,makecell}
\usepackage{longtable}
\usepackage{booktabs}
\usepackage{tabularx}
\usepackage{setspace}
%------------------------------------------------------------------------------%


%---------------------------------添加列表控制---------------------------------%
\usepackage{enumerate}
\usepackage{enumitem}
%------------------------------------------------------------------------------%


%---------------------------------设置引用格式---------------------------------%
\renewcommand\figureautorefname{图}
\renewcommand\tableautorefname{表}
\renewcommand\equationautorefname{式}
\newcommand\myreference[1]{[\ref{#1}]}
\newcommand\eqrefe[1]{式(\ref{#1})}
\renewcommand\theequation{\thechapter.\arabic{equation}}
% 增加 \ucite 命令使显示的引用为上标形式
\newcommand{\ucite}[1]{$^{\mbox{\scriptsize \cite{#1}}}$}
%------------------------------------------------------------------------------%


%--------------------------设置中文段落缩进与正文版式--------------------------%
\XeTeXlinebreaklocale "zh"       %使用中文的换行风格
\XeTeXlinebreakskip = 0pt plus 1pt    %调整换行逻辑的弹性大小
%------------------------------------------------------------------------------%

%----------------------------设置段落标题与目录格式----------------------------%
\setcounter{secnumdepth}{4}
\setcounter{tocdepth}{4}


% 正文中标题格式,毋需标号
% \titleformat{\section}[hang]{\fHei \sf \sSihao}
%     {\sSihao }{0.5em}{}{}
% \titleformat{\subsection}[hang]{\fHei \sf \sHalfXiaosi}
%     {\sHalfXiaosi }{0.5em}{}{}
% \titleformat{\subsubsection}[hang]{\fHei \sf}
%     {\thesubsubsection }{0.5em}{}{}
% 正文中标题格式,需要标号

\newcommand\chapterID[1]{第\CJKnumber{#1}章}
\renewcommand{\chaptername}{第\CJKnumber{\thechapter}章}
\renewcommand{\figurename}{图}
\renewcommand{\tablename}{表}
\renewcommand{\bibname}{参考文献}
\renewcommand{\contentsname}{目~录}


\titleformat{\chapter}[hang]{\normalfont\sErhao\filcenter\fHei\bf}{\sErhao{\chaptertitlename}}{20pt}{\sErhao}
\titleformat{\section}[hang]{\fHei \bf \sSanhao}{\sSanhao \thesection}{0.5em}{}{}
\titleformat{\subsection}[hang]{\fHei \bf \sSihao}{\sSihao \thesubsection}{0.5em}{}{}
\titleformat{\subsubsection}[hang]{\fHei \bf}{\thesubsubsection }{0.5em}{}{}


% 缩小正文中各级标题之间的缩进
\titlespacing{\chapter}{0pt}{-3ex plus .1ex minus .2ex}{0.25em}
\titlespacing{\section}{0pt}{-0.2em}{0em}
\titlespacing{\subsection}{0pt}{0.5em}{0em}
\titlespacing{\subsubsection}{0pt}{0.25em}{0pt}

% 定义目录中各级标题之间的格式以及缩进
% \dottedcontents{chapter}[0.0em]{\fHei\vspace{0.5em}}{0.0em}{5pt}
% \dottedcontents{section}[1.16cm]{}{1.8em}{5pt}
% \dottedcontents{subsection}[2.00cm]{}{2.7em}{5pt}
% \dottedcontents{subsubsection}[2.86cm]{}{3.4em}{5pt}
% \titlecontents{chapter}[0pt]{\fHei\vspace{0.5em}}%
%     {\contentsmargin{0pt}\fHei\makebox[0pt][l]{\chapterID{\thecontentslabel}}\hspace{3.8em}}%
%     {\contentsmargin{0pt}\fHei}%
%     {}[\vspace{0em}]


%------------------------------------------------------------------------------%

%---------------------------------设置页眉页脚---------------------------------%
\usepackage{fancyhdr}
\usepackage{fancyref}
\pagestyle{fancy}
\cfoot{}
\fancyfoot[RO, LE]{\thepage}
\fancyfoot[LO]{{\tt \artversion}}
\fancyfoot[RE]{{\it \shorttitle}}
\fancypagestyle{plain}{
    \fancyhf{}
    \fancyfoot[RO, LE]{\thepage}
    \fancyfoot[LO]{{\tt \artversion}}
    \fancyfoot[RE]{{\it \shorttitle}}
}

\renewcommand{\headrulewidth}{0.4pt}
\renewcommand{\headwidth}{\textwidth}
\renewcommand{\footrulewidth}{0pt}
%------------------------------------------------------------------------------%


%-------------------------------数学符号格式控制-------------------------------%
\usepackage{bm}
\def\mathbi#1{\textbf{\em #1}}
% Caligraphic letters:      \mathcal{A}
% Mathbb letters:           \mathbb{A}
% Mathfrak letters:         \mathfrak{A}
% Math Sans serif letters:  \mathsf{A}
% Math bold letters:        \mathbf{A}
% Math bold italic letters: \mathbi{A}
%------------------------------------------------------------------------------%


%-------------------------------数学特殊符号控制-------------------------------%
%-------------------------------数学特殊符号控制-------------------------------%
\newcommand\mi{{\mathrm i}}             % constant i
\newcommand\me{{\mathrm e}}             % constant e
\newcommand\mreal{{\mathbb R}}          % constant real set R
\newcommand\mhilb{{\mathbb H}}          % constant hilbert set H
\newcommand\mcond{{\mathrm{Cond.}}}     % condition symbol
\newcommand\mconst{{\mathrm{const}}}    % constant symbol
%------------------------------------------------------------------------------%
\newcommand\mva{{\bm a}}                % vector a
\newcommand\mvb{{\bm b}}                % vector b
\newcommand\mvc{{\bm c}}                % vector c
\newcommand\mvd{{\bm d}}                % vector d
\newcommand\mve{{\bm e}}                % vector e
\newcommand\mvf{{\bm f}}                % vector f
\newcommand\mvg{{\bm g}}                % vector g
\newcommand\mvh{{\bm h}}                % vector h
\newcommand\mvi{{\bm i}}                % vector i
\newcommand\mvj{{\bm j}}                % vector j
\newcommand\mvk{{\bm k}}                % vector k
\newcommand\mvl{{\bm l}}                % vector l
\newcommand\mvm{{\bm m}}                % vector m
\newcommand\mvn{{\bm n}}                % vector n
\newcommand\mvo{{\bm o}}                % vector o
\newcommand\mvp{{\bm p}}                % vector p
\newcommand\mvq{{\bm q}}                % vector q
\newcommand\mvr{{\bm r}}                % vector r
\newcommand\mvs{{\bm s}}                % vector s
\newcommand\mvt{{\bm t}}                % vector t
\newcommand\mvu{{\bm u}}                % vector u
\newcommand\mvv{{\bm v}}                % vector v
\newcommand\mvw{{\bm w}}                % vector w
\newcommand\mvx{{\bm x}}                % vector x
\newcommand\mvy{{\bm y}}                % vector y
\newcommand\mvz{{\bm z}}                % vector z
%------------------------------------------------------------------------------%
\newcommand\mvzero{{\bm 0}}             % vector 0
\newcommand\mvone{{\bm 1}}              % vector 1
\newcommand\mvalpha{{\bm \alpha}}       % vector alpha
\newcommand\mvomega{{\bm \omega}}       % vector omega
\newcommand\mvtheta{{\bm \theta}}       % vector theta
%------------------------------------------------------------------------------%
\newcommand\mma{{\mathbf A}}            % matrix A
\newcommand\mmb{{\mathbf B}}            % matrix B
\newcommand\mmc{{\mathbf C}}            % matrix C
\newcommand\mmd{{\mathbf D}}            % matrix D
\newcommand\mme{{\mathbf E}}            % matrix E
\newcommand\mmf{{\mathbf F}}            % matrix F
\newcommand\mmg{{\mathbf G}}            % matrix G
\newcommand\mmh{{\mathbf H}}            % matrix H
\newcommand\mmi{{\mathbf I}}            % matrix I
\newcommand\mmj{{\mathbf J}}            % matrix J
\newcommand\mmk{{\mathbf K}}            % matrix K
\newcommand\mml{{\mathbf L}}            % matrix L
\newcommand\mmm{{\mathbf M}}            % matrix M
\newcommand\mmn{{\mathbf N}}            % matrix N
\newcommand\mmo{{\mathbf O}}            % matrix O
\newcommand\mmp{{\mathbf P}}            % matrix P
\newcommand\mmq{{\mathbf Q}}            % matrix Q
\newcommand\mmr{{\mathbf R}}            % matrix R
\newcommand\mms{{\mathbf S}}            % matrix S
\newcommand\mmt{{\mathbf T}}            % matrix T
\newcommand\mmu{{\mathbf U}}            % matrix U
\newcommand\mmv{{\mathbf V}}            % matrix V
\newcommand\mmw{{\mathbf W}}            % matrix W
\newcommand\mmx{{\mathbf X}}            % matrix X
\newcommand\mmy{{\mathbf Y}}            % matrix Y
\newcommand\mmz{{\mathbf Z}}            % matrix Z
%------------------------------------------------------------------------------%
\newcommand\mmlambda{{\bm \Lambda}}     % matrix \Lambda
%------------------------------------------------------------------------------%
\DeclareMathOperator\diff{d\!}          % operator diff
\DeclareMathOperator\Diff{D\!}          % operator Diff
\DeclareMathOperator\Expect{E}          % operator Expect
\DeclareMathOperator\diag{diag}         % operator diag
\DeclareMathOperator\eig{eig}           % operator eig
\DeclareMathOperator\lcm{lcm}           % operator lcm
\DeclareMathOperator\rand{rand}         % operator rand
\DeclareMathOperator\tr{tr}             % operator tr
\DeclareMathOperator\var{var}           % operator var
\DeclareMathOperator\cov{cov}           % operator cov
\DeclareMathOperator\mean{mean}         % operator mean
\DeclareMathOperator*{\argmin}{argmin}  % operator argmin
\DeclareMathOperator*{\argmax}{argmax}  % operator argmax
\DeclareMathOperator{\dist}{dist}       % operator dist
\allowdisplaybreaks[4]
%------------------------------------------------------------------------------%
\newcommand\mmat{{{\mathbf A}^T}}       % matrix A^T
\newcommand\mmbt{{{\mathbf B}^T}}       % matrix B^T
\newcommand\mmct{{{\mathbf C}^T}}       % matrix C^T
\newcommand\mmdt{{{\mathbf D}^T}}       % matrix D^T
\newcommand\mmet{{{\mathbf E}^T}}       % matrix E^T
\newcommand\mmft{{{\mathbf F}^T}}       % matrix F^T
\newcommand\mmgt{{{\mathbf G}^T}}       % matrix G^T
\newcommand\mmht{{{\mathbf H}^T}}       % matrix H^T
\newcommand\mmit{{{\mathbf I}^T}}       % matrix I^T
\newcommand\mmjt{{{\mathbf J}^T}}       % matrix J^T
\newcommand\mmkt{{{\mathbf K}^T}}       % matrix K^T
\newcommand\mmlt{{{\mathbf L}^T}}       % matrix L^T
\newcommand\mmmt{{{\mathbf M}^T}}       % matrix M^T
\newcommand\mmnt{{{\mathbf N}^T}}       % matrix N^T
\newcommand\mmot{{{\mathbf O}^T}}       % matrix O^T
\newcommand\mmpt{{{\mathbf P}^T}}       % matrix P^T
\newcommand\mmqt{{{\mathbf Q}^T}}       % matrix Q^T
\newcommand\mmrt{{{\mathbf R}^T}}       % matrix R^T
\newcommand\mmst{{{\mathbf S}^T}}       % matrix S^T
\newcommand\mmtt{{{\mathbf T}^T}}       % matrix T^T
\newcommand\mmut{{{\mathbf U}^T}}       % matrix U^T
\newcommand\mmvt{{{\mathbf V}^T}}       % matrix V^T
\newcommand\mmwt{{{\mathbf W}^T}}       % matrix W^T
\newcommand\mmxt{{{\mathbf X}^T}}       % matrix X^T
\newcommand\mmyt{{{\mathbf Y}^T}}       % matrix Y^T
\newcommand\mmzt{{{\mathbf Z}^T}}       % matrix Z^T
%------------------------------------------------------------------------------%

\endinput
% 这是所有常用的数学符号的导言区设置,不能单独编译。
% by ruby script
% f = File.new("math-symbols.tex", 'w+')
% mv, mm, mt = [], [], []
% ('a'..'z').each { |c|
% 	mv << "\\newcommand\\mv#{c}{{\\bm #{c}}}                % vector #{c}\n"
% 	mm << "\\newcommand\\mm#{c}{{\\mathbf #{c.upcase}}}            % matrix #{c.upcase}\n"
% 	mt << "\\newcommand\\mm#{c}t{{{\\mathbf #{c.upcase}}^T}}       % matrix #{c.upcase}^T\n"
% }
% mv.each { |s| f.write(s) }
% mm.each { |s| f.write(s) }
% mt.each { |s| f.write(s) }
% f.close()
%------------------------------------------------------------------------------%


%---------------------------------添加列表控制---------------------------------%
\usepackage{enumerate}
\numberwithin{equation}{section}
\renewcommand\theequation{\thesection.\arabic{equation}}
%------------------------------------------------------------------------------%


%---------------------------------添加代码控制---------------------------------%
\usepackage{listings}
\lstset{
    basicstyle=\footnotesize\ttfamily,
    numbers=left,
    numberstyle=\tiny,
    numbersep=5pt,
    tabsize=4,
    extendedchars=true,
    breaklines=true,
    keywordstyle=\color{blue},
    numberstyle=\color{purple},
    commentstyle=\color{olive},
    stringstyle=\color{orange}\ttfamily,
    showspaces=false,
    showtabs=false,
    framexrightmargin=5pt,
    framexbottommargin=4pt,
    showstringspaces=false,
    escapeinside=`', %逃逸字符(1左面的键),用于显示中文
}
\renewcommand{\lstlistingname}{CODE}
\lstloadlanguages{% Check Dokumentation for further languages, page 12
    Pascal, C++, Java, Ruby, Python, Matlab, R
}
%------------------------------------------------------------------------------%

\endinput
%这是简单的 article 的导言区设置,不能单独编译。